\chapter{La soluzione - Design e Implementazione}

\section{Il linguaggio: Go}


\section{Il sistema Warehouse}

\subsection{Il sistema di code: NSQ}

\subsection{Il database: TimescaleDB}

\subsection{Il servizio di storage: MinIO}


\section{Realizzazione delle API}

In collaborazione con Gargano Daniele e Provornyy Igor sono state ideate e realizzate delle API per permettere agli utenti di ottenere i dati di cui alla sezione \ref{parametri}. Le API di seguito presentate sono le API su cui il lavoro che si sta mostrando si incentra maggiormente, per cui non saranno elencate e descritte le restanti.

\subsection{Il formato YAML}

\subsection{API byLocation}

\subsection{API byDevices}

\section{Processing delle richieste}

\subsection{Estrazione e Parsing}

La richiesta, in formato JSON, viene estratta da NSQ e "tradotto" in una struttura accuratamente ideata per rappresentare la richiesta.

\subsection{Interrogazione del database}

\subsection{Zip dei file}

\subsection{Upload su MinIO}