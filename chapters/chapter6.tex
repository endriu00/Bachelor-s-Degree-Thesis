\chapter{Ringraziamenti}
Ho pensato spesso a cosa avrei scritto in questa sezione, cosa mi avrebbero suggerito il cuore e la mente in questo momento, quale dei due avrei seguito con maggior fedeltà. \\ \\ 
Non posso dire che questi tre anni siano stati semplici, per molti motivi. La facoltà che ho scelto mi ha sicuramente posto davanti a sfide sempre maggiori, che sono sempre riuscito ad affrontare, mettendomi in dubbio, spesso, cercando di migliorare, sempre, per passione e caparbietà. L'informatica è sempre stata un punto cardine per me, un mondo che ha sempre e sempre susciterà il mio interesse, che sarà sempre affascinante ed ineguagliabile. Lo capisco mentre ne parlo con chi ne sa di meno, con chi ne sa di più. \\ \\ 
Il tempo ha fatto il suo corso, imprescindibile, portando con sé persone a me care. Penso ai miei nonni, a mio zio, andati via in così poco tempo, così velocemente. Per non parlare della situazione nel mondo, della pandemia, delle quarantene: è come se una parte dell'esistenza fosse stata messa in pausa, come se la vita di molti di noi fosse stata messa in pausa. O, perlomeno, la mia. \\ \\ 
All'improvviso, però, in un momento irrintracciabile del tempo, esso torna a fluire. Il passato non sembra altro che passato, il futuro qualcosa di incerto da scrivere, il presente il momento in cui si vive. E, così, guardando indietro verso questi tre anni appena trascorsi, tutto assume un connotato nostalgico, una visione diversa di ciò che è stata. E ripenso a tutte le cose che invece sono state semplici, alle persone che li hanno resi indimenticabili, irrinunciabili. \\ \\ 
Voglio, a tal proposito, ringraziare innanzitutto il mio relatore, il prof. Emanuele Panizzi, cha ha contribuito alla formazione della mia passione per questo mondo con il corso che ho potuto seguire prima, con il tirocinio proposto poi. Rimanendo in questo contesto, voglio anche ringraziare il dottorando Enrico Bassetti, per essere stato sempre a disposizione per ogni dubbio, per avermi spinto a voler imparare sempre di più e per avermi insegnato moltissimo.
\\ \\ \\ \\
Voglio ringraziare la persona che mi ha sempre supportato, condiviso ogni ricordo, ogni momento, ogni situazione piacevole e spiacevole, Chiara. Non ho parole per descriverti. Sei sempre la mia forza interiore, il motivo per cui non posso mai darmi per vinto, e spero di ripagarti almeno in parte per tutto ciò che sei per me. A volte ti guardo pensando a come riesci ad essere così speciale, così unica. E penso che un grande merito va alla tua famiglia, che ringrazio per essere un punto di riferimento costante per me.\\ \\ 
Ringrazio la mia famiglia, papà per quei momenti padre e figlio non troppo convenzionali, come le "lezioni" su come progettare un impianto elettrico domestico, mamma per quell'ottimismo e quei sogni che, implicitamente, mi hanno sempre spinto a pensare in grande, a guardarmi intorno, ed i miei fratelli, che mi hanno insegnato tanto ed i cui consigli porterò sempre con me; penso alle conversazioni mai banali con Simone, alle sempre nuove idee di Alessandro, ai consigli paterni di Francesco, alle tante cose che ho da imparare anche da Leonardo, il più piccolo di noi; penso ai tanti nipoti(ni), alle nuore, alle cose belle che accadono, come la nascita di Federico, i momenti tutti assieme. \\ \\ 
Ringrazio i miei cugini, reali e "acquisiti" (Pami, Alessio, Stefano), nonna Anna, gli zii, i miei amici, a partire dai più stretti (è doveroso citare Giovanni, Filippo, Irene, Alessio, Giulia, Alessia, Alberto, Alice, Giuseppe, Eleonora, Mattia, Cassandra, Luigi) a quelli che non vedo più così spesso, a quelli che ho potuto conoscere solamente nell'ultimo anno dietro uno schermo. \\ \\ 
Ringrazio ogni persona che ho avuto il piacere di conoscere per quanto mi ha insegnato, per avermi fatto capire che c'è sempre qualcosa di nuovo da imparare, che è un po' il dogma di noi informatici. Ringrazio chi ha reso la mia vita più semplice o più difficile, perché non sempre le lezioni si imparano nel migliore dei modi. Ringrazio gli attimi da solo, i momenti di riflessione, il tempo trascorso da solo, perché non si può essere veramente felici se non si è felici con se stessi. E, infine, ringrazio chi mi ha insegnato che non sempre le cose vanno per il verso giusto; o che, a volte, il verso giusto non è rivolto verso di noi. E che, a volte, bisogna anche spostarsi per allinearsi ad esso.\\ \\
Tirando le somme del tempo trascorso non sempre è possibile essere grati per ciò che è stato: è più facile condannare il passato e rifugiarsi nel futuro piuttosto che accettarlo e rivalutarlo, cercando di ricordare cosa più di bello è stato.
Mi chiedo spesso se rifarei tutto ciò che ho fatto, se rivivrei tutto ciò che ho vissuto, se sceglierei ancora questo percorso che ho iniziato tre anni fa. Se tutto il sudore versato sarà mai ripagato. Se tutte le lezioni imparate saranno servite a qualcosa. \\ \\ La risposta è sempre sì. Finché posso contare su tutte queste fantastiche persone, la risposta sarà sempre sì.
